% vim: set textwidth=72 :
% Copyright (C) 2013 Roy E Lowrance
% See the file COPYING.txt for copying conditions.

\documentclass{article}
\usepackage{amssymb,latexsym,amsmath}
\usepackage{upquote}
\let\code\texttt % in line source code

\begin{document}
\title{Installing Torch}
\author{Roy E Lowrance}
\maketitle

%\abstract{XXX}

\section{Problem}
You want to install torch version 7.

\section{Solution}

The solution depends on the operating system you are using. Choose the
closest match, though the recipes are tested only on the specified OS
versions.

\subsection{Ubuntu 12.04}

If you are using Ubuntu 12.04, open a new terminal and install the curl
program. Enter your password when prompted.

\begin{verbatim}
$ sudo apt-get install curl
\end{verbatim}

Now install torch by entering the command below. The command should
appear entirely on one line. The line is broken up below so that it will
print nicely.

\begin{verbatim}
curl -s
  https://raw.github.com/clementfarabet/torchinstall/master/install-all | 
  PREFIX=~ bash
\end{verbatim}

The character after \code{PREFIX=} is a tilde.

If nothing happens after entering the \code{curl} command. enter it
again without the \code{-s} option, which means silent and surpresses
curl's messages.

If you are prompted to enter your password, do so.

You should see many message scroll by and eventually a message that ``Torch7 has
been installed successfully.''

You can verify that torch is installed by listing your \code{bin} directory

\begin{verbatim}
$ ls ~/bin
\end{verbatim}

You should see an executable for \code{torch} and other files as well.

Print and examine the bash variable PATH.

\begin{verbatim}
$ echo path=$PATH
\end{verbatim}

If your PATH starts with \code{\textasciitilde/bin} skip down to the
paragraph that says to open a new terminal. Otherwise, you need to 
tell bash that your preferred programs are in your
\code{\textasciitilde/bin} directory. Do so by entering these commands:

\begin{verbatim}
$ echo 'source ~/.bashrc' >> ~/.bash_profile
$ echo PATH=$HOME/bin:$PATH >> ~/.bashrc
$ source ~/.bash_profile
$ which torch
\end{verbatim}

After the last line, you should see that bash will use the torch in your
\code{\textasciitilde/bin} directory.

Open a new terminal, start torch, and enter these commands:

\begin{verbatim}
$ torch
<torch prints some stuff>
t7> = 1 + 2 <ENTER>
\end{verbatim}

Torch should print \code{3}.

Exit torch by entering c-D (hold down the control key and type the
letter ``d'') and typing \code{y}, if prompted to do so.

\subsection{OS X}

\subsubsection{OS X 10.8.5 Mountain Lion}

Clement, the author of the one line installer,  uses a late release of OS X and is the author of the one-line
installer. 

Unbuntu comes with a compiler and numerous command line tools already
installed. OS X does not. The first steps are to install the missing
sofware: a compiler and command line tools.   

Apple's compiler is called Xcode. Navigate to
\code{https://developer.apple.com/xcode/} and install  the latest
version of Xcode 4.6. (I installed 4.6.3.). After the install of Xcode,
run it and install any system components that it suggests.

To install the command line tools, go to
\code{develop.apple.com/downloads} and search for command line tools.
Pick the right version for your operating system. Download the .dmg file
and install it.

Now follow the directions for Ubuntu.



\subsubsection{OS X 10.6.8 Snow Leopard}


You need to install Xcode 3.2.6 (not a 4.x.y version). Search
\code{developer.apple.com} for Xcode 3.2.6. Download it and install.
This download includes the command line tools.

Now follow the directions for Ubuntu.

The one-line installer does not work. I don't know of a work around.

\subsection{Windows}

If you are using Windows, this recipe solution will not work. The
developers of torch claim that torch itself will work under Windows but
that some of its preferred libraries do not yet work under Windows.
Windows users might consider installing Virtual Box and using it to
create a Ubuntu 12.04 64-bit virtual machine. Then the above solution
can be used to install torch into the virtual machine.

\section{Discussion}

The solution uses the one-line torch install from Clement Farabet. one
of the torch developers. His GitHub site is at
\code{https://github.com/clementfarabet/}. There you will find much
useful torch code including his one-line install.


If you want to install torch on servers where you do not have root
access, you will need to install torch locally. Here ``locally'' means
that the torch instance will be local to one account. Not having root
access is a common use case so the above recipe suggests always
installing torch locally, so that your torch installation is always in
the same place if you are running on your own system or a server.

In addition to installing into \code{\textasciitilde/bin}, the one-line
installer puts files into \code{\textasciitilde/etc},
\code{\textasciitilde/include}, \code{\textasciitilde/lib}, and
\code{\textasciitilde/share}. For this reason, some people prefer to
install torch locally into \code{\textasciitilde/local} so that these
additional directories are not modified by the install.  If you like that
choice, re-run the one-line install with
\code{PREFIX=\textasciitilde/local} and adjust your \code{.bashrc} file.
All the files are now installed in the \code{\textasciitilde/local}
directory. In particulary, the torch executables are in
\code{\textasciitilde/local/bin}.
Don't forget to delete the torch executables in your
\code{\textasciitilde/bin} directory.

The solution installs torch as well as some libraries that torch can use
and some torch packages. When you are a beginner, you definitely want
that as the packages that are installed are frequently used and the
non-torch libraries are used by torch to run fast. However,you may not
want that. For example, you may have space restrictions on the device
where you are installing torch. Or you may want to not use some of the
libraries so that you can benchmark torch versus other technology that
does not use those libraries. Or you may want to experiment with other
libraries. How to install torch in a custom way can be found at
\code{http://www.torch.ch/manual/install/index}. 

The ``7'' in ``torch7'' refers to the version of
torch. There was a version 5 that is very different from version 7.

The PATH is shell variable that is set to a colon-separated list of
directories where bash looks for an executable. The search stops in the
first directory where the executable is found. If the executable is not
in any of the directories, you will get an error message: \code{command
not found}.  \section{See also}

This recipe is free documentation. You can modify it by visiting the
github for account ``rlowrance'' and forking the repo
``torch-cookbook.''

\end{document}


