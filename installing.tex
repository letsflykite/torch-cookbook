% vim: set textwidth=72 
% Copyright (C) 2013 Roy E Lowrance
% See the file LICENSE.txt for copying conditions.

\documentclass{article}
\usepackage{amssymb,latexsym,amsmath}
\let\code\texttt % in line source code

\begin{document}
\title{Installing Torch}
\author{Roy E Lowrance}
\maketitle

%\abstract{XXX}

\section{Problem}
You want to install torch version 7.

\section{Solution}
If you are using OS X 10.8 or Ubuntu 12.04, navigate to\\
\code{https://github.com/clementfarabet/torchinstall}, read the
documentation on installing torch, and execute this command (which
should be typed on one line in a terminal)\\
\code{curl -s
  https://raw.github.com/clementfarabet/torchinstall/master/install-torch
| PREFIX=\textasciitilde/local bash}

You should see a message that ``Torch7 has been installed
successfully.''

You now need to set your path so that torch and other local programs can
be found. To do that, edit your \code{\textasciitilde/.bashrc} file to
include the line\\
\code{PATH=\textasciitilde/local/bin:\$PATH}\\
and execute the lines in the file by entering in the terminal\\
\code{source \textasciitilde/.bashrc}\\
so that your terminal knows about the PATH variable.

If you are using Windows, this recipe solution will not work. The
developers of torch claim that torch itself will work under Windows but
that some of its preferred libraries do not yet work under Windows.
Windows users might consider installing Virtual Box and using it to
create a Ubuntu 12.04 64-bit virtual machine. Then the above solution
can be used to install torch into the virtual machine.

\section{Discussion}

To check if torch is really installed, you can execute \code{torch}
 at your command prompt. You
should see the torch 7 prompt \code{t7>}. Enter ``=1 +2'' and press
ENTER. Torch should print ``3'' and give you another prompt. This time
enter c-D (hold down the control key and press the lower case d key). If
torch asks if it should really quit, press ``y'' and ENTER. You should
be back at the command prompt.

If you can't get torch to start using the \code{torch} command, trying
entering the command \code{\textasciitilde/local/bin/torch}. It should
work as described above. If it works and \code{torch} by itself does
not, something has gone wrong with your \code{PATH} variable. Check the
changes you made to your \code{\textasciitilde/.bashrc} file.


The recommended solution is to install torch locally using Clement
Farabet's one-line installer. It works only on OS X and
Ubunutu. At the time this article was written, it had been tested on OS
X 10.8 and Ubuntu 12.04. It may work on other releases of these and
similar operating systems.

If you want to install torch on servers where you do not have root
access, you will need to install torch locally. Here ``locally'' means
that the torch instance will be local to one account. Not having root
access is a common use case so the above recipe suggests always
installing torch locally, so that your torch installation is always in
the same place if you are running on your own system or a server.

Where should you install torch locally? Any directory in your home
directory will work. We recommen installing torch in a common place. The
``Filesystem Hierarchy Standard (FHS)'' (see \\
\code{en.wikipedia.org/wiki/Filesystem\_Hierarchy\_Standard} accessed
September 12, 2013) states that programs like torch when installed
globally (for all users) should go into sub-directories of the directory
\code{/usr/local/}. We suggest following this practice and putting torch
into \code{/home/ID/local/} where \code{ID} is your user id.  

The solution installs torch as well as some libraries that torch can use
and some torch packages. You may not want that. For example, you may
have space restrictions on the device where you are installing torch. Or
you may want to not use some of the libraries so that you can benchmark
torch versus other technology that does not use those libraries. Or you
may wnat to experiment with other libraries. How to
install torch in a custom way can be found at
\code{http://www.torch.ch/manual/install/index}. 

The ``7'' refers to the version of
torch. There was a version 5 that is very different from the version 7.

\section{See also}

Other recipees of interest to new users of torch include these:
\begin{itemize}
  \item Reading the torch documentation
  \item Starting up
  \item Torch as a calculator
  \item Numbers
  \item Variables
  \item Expressions
  \item Trig functions
  \item Elementary functions
  \item Tensors
  \item Reading files
  \item Writing files
  \item Linear algebra
  \item Element-wise operations
  \item Statements
\end{itemize}



\end{document}


