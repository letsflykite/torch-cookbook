% vim: set textwidth=72 :
% Copyright (C) 2013 Roy E Lowrance
% See the file COPYING.txt for copying conditions.

\documentclass{article}
\usepackage{amssymb,latexsym,amsmath}
\usepackage{verbatim}
\let\code\texttt % in line source code

\begin{document}
\title{Run Many Hadoop Jobs}
\author{Roy E Lowrance}
\maketitle

%\abstract{XXX}

\section{Problem}

You want to run many Hadoop jobs without writing a long script for each.

\section{Solution}

Use the script \code{map-reduce.sh} which is invoked this way:

\begin{verbatim}
$ ./map-reduce.sh INPUT JOB [READLIMIT]
\end{verbatim}

where
\begin{itemize}
  \item INPUT is the path to the input file in the Hadoop file system
  \item JOB is the name of the map-reduce job. The mapper is
    \code{JOB-map.lua}.  The reducer is \code{JOB-reduce.lua}
  \item READLIMIT is ignored. It is provided for compatiblity with the 
    \code{map-reduce-local.sh} script which is described in the recipe ``I
    want to test my Hadoop program.''
\end{itemize}

The \code{map-reduce.sh} script will run your job and copy the job's
output directory to the local file system in directory
\code{\$HOME/map-reduce-output/}. You can change this directory by
editing the script.

I like to display at least the first portion of the output file. I do
this by writing another script \code{go.sh} which runs \code{map-reduce.sh} then
prints the first output file.

\section{Discussion}

The script \code{map-reduce.sh} is just below. It is a modification of
the script presented in ``Using Torch on Hadoop.'' Refer to that recipe
for documentation.

\verbatiminput{run-many-hadoop-jobs-assets/map-reduce.sh}

The script \code{go.sh} is just below. Modify it so that it uses your
INPUT and JOB and prints what is relevant to your work.


\verbatiminput{run-many-hadoop-jobs-assets/go.sh}



\section{See also}

Other relevant recipes include
\begin{itemize}
  \item ``Using Torch on Hadoop''
  \item ``I want to test my Hadoop program''
\end{itemize}

This recipe is free documentation. You can modify it by visiting
github for account ``rlowrance'' and forking the repo
``torch-cookbook.''

\end{document}


